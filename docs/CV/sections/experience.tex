\documentclass[letter,10pt]{article}
\usepackage{TLCresume}
\begin{document}

\subsection{MIT Media Lab}
\subsection{{AstroAnt --- A miniature robotic swarm for servicing on in-orbit spacecraft external surfaces\hfill Jan 2021 --- Present}}
\begin{zitemize}
\item Designed two different kinds of autonomous miniature wheeled robotic swarms for servicing on the external surfaces of in-orbit spacecrafts. The robots are equipped with magnetic wheels and can move on magnetic surfaces. With a modular design, each robot can carry different sensor payloads and perform inspection sensing to help with in-orbit maintenance.
\item Finished four zero-gravity flights to test the mobility and sensing capabilities of the AstroAnt in zero gravity and lunar gravity environment. The work reached Technology Readiness Level 6.
\item One AstroAnt robot will be sent to the Lunar south pole with the MAPP-1 rover developed by Lunar outpost between the Dec 2022 --- Jan 2023 .
\end{zitemize}

\subsection{{LunarWSN --- a Wireless Sensor Network node designed for In-Situ lunar water ice detection \hfill Nov 2020 --- Aug 2021}}
\begin{zitemize}
\item Designed a fully functional cubic sensor node prototype that can be ballistically deployed from a rover or lander to regions of interest that might be unsafe or impractical for rovers or landers. The node is a light (<170g), miniaturized (5cm x 5cm x 5cm), modular design, that allows sensor payloads to be customized to different scientific missions. As a representative case study, the node is equipped with an impedance sensor designed to measure the permittivity of the lunar soil, which infers water content.
\item Finished the system function tests (wireless localization, wireless communication, and sensing capability) in a lab environment. The work reached Technology Readiness Level 4.
\end{zitemize}

\subsection{{WOSNA --- Work Out on-body Sensor Network Assistant \hfill Aug 2020 --- Present}}
\begin{zitemize}
\item Designed an on-body sensor network that monitors workout performance. The sensor network is composed of multiple miniature sensor nodes, and each node is a tiny suction cup equipped with a pulse sensor and Bluetooth Low Energy communication. The sensor nodes can suck on the desired part of the body to monitor the performance of some specific muscle. The sensor nodes are very small and can adapt to the irregular body surface.
\end{zitemize}

\subsection{{PCBPT --- PCB automatic probe tester \hfill Aug 2019 --- Oct 2019}}
\begin{zitemize}
\item Designed an automatic PCB probe system to help with PCB debugging.
With the help of PCBPT, users can choose desired signals in the schematic,
and the PCBPT will choose proper pads on the PCB for the selected signals
and place probes on the pads. All the users need to do are selecting signals
and checking the waveforms on an oscilloscopes.
\end{zitemize}

\subsection{CERN (the European Organization for Nuclear Research)}
\subsection{{AMS-02 UTTPS ground test monitor and control software\hfill Apr 2018 --- Apr 2019}}
\begin{zitemize}
\item Developed the control and monitoring software for the thermal vacuum test of the AMS-02 (Alpha Magnetic Spectrometer, a state-of-the-art particle physics detector operating on the International Space Station) Upgraded Tracker Thermal Pump System (UTTPS). The software is designed by using labVIEW.
\item The UTTPS has been installed to the AMS-02 by the end of Jan 2020 through four space walks.
\end{zitemize}

\end{document}