%%%%%%%%%%%%%%%%%%%%%%%%%%%%%%%%%%%%%%%%%
% Medium Length Professional CV
% LaTeX Template
% Version 2.0 (8/5/13)
%
% This template has been downloaded from:
% http://www.LaTeXTemplates.com
%
% Original author:
% Trey Hunner (http://www.treyhunner.com/)
%
% Important note:
% This template requires the resume.cls file to be in the same directory as the
% .tex file. The resume.cls file provides the resume style used for structuring the
% document.
%
%%%%%%%%%%%%%%%%%%%%%%%%%%%%%%%%%%%%%%%%%

%----------------------------------------------------------------------------------------
%   PACKAGES AND OTHER DOCUMENT CONFIGURATIONS
%----------------------------------------------------------------------------------------

\documentclass{resume} % Use the custom resume.cls style
\usepackage[left=0.7in,top=0.4in,right=0.7in,bottom=0.5in]{geometry} % Document margins
\usepackage{hyperref}
%----------------------------------------------------------------------------------------
%   DOCUMENT START
%----------------------------------------------------------------------------------------
\name{Fangzheng Liu} % Your name
\address{\href{mailto:fzliu@mit.edu}{fzliu@mit.edu} \\ \href{https://github.com/Ironman-007}{Fangzheng's github} \\ \href{https://ironman-007.github.io/FZ-portfolio/index.html}{Portfolio}} % Your phone number and email
\begin{document}
\title{Fangzheng Resume}
%----------------------------------------------------------------------------------------
%   EDUCATION SECTION
%----------------------------------------------------------------------------------------

\begin{rSection}{Education}

\begin{rSubsection}{Massachusetts Institute of Technology}{September 2021 --- Present}{Ph.D. in Media Arts and Sciences}{Responsive Environments Group, MIT Media Lab}
%\vspace{-.1cm}
%Overall GPA: 78\%
\vspace{-.65cm}
\item[]
\end{rSubsection}

\begin{rSubsection}{Massachusetts Institute of Technology}{September 2019 --- Aug 2021}{M.S. in Media Arts and Sciences}{Responsive Environments Group, MIT Media Lab}
%\vspace{-.1cm}
%Overall GPA: 78\%
\vspace{-.65cm}
\item[]
\end{rSubsection}

\begin{rSubsection}{Beijing Institute of Technology}{September 2015 --- Apr 2018}{M.S. in Information and Communication Engineering}{}
%\vspace{-.1cm}
%Overall GPA: 78\%
\vspace{-.65cm}
\item[]
\end{rSubsection}

\begin{rSubsection}{Beijing Institute of Technology}{September 2011 --- Jun 2015}{B.S. in Information Engineering}{}
%\vspace{-.1cm}
%Overall GPA: 78\%
\vspace{-.65cm}
\item[]
\end{rSubsection}

\end{rSection}

%----------------------------------------------------------------------------------------
%   WORK EXPERIENCE SECTION
%----------------------------------------------------------------------------------------

\begin{rSection}{ACADEMIC RESEARCH EXPERIENCE}
\begin{rSubsection}{AstroAnt}{Jan 2021 --- Present}{ }{MIT Media Lab}
\item The AstroAnt is a miniature robotic swarm for servicing on in-orbit spacecraft external surfaces.
\item Designed two different kinds of autonomous miniature wheeled robotic swarms for servicing on the external surfaces of in-orbit spacecraft. The robots are equipped with magnetic wheels and can move on magnetic surfaces. With a modular design, each robot can carry different sensor payloads and perform inspection sensing to help with in-orbit maintenance.
\item Finished four zero-gravity flights to test the mobility and sensing capabilities of the AstroAnt in zero gravity and lunar gravity environment. The work reached Technology Readiness Level 6.
\item One AstroAnt robot will be sent to the Lunar south pole with the MAPP-1 rover developed by the Lunar Outpost around June 2023.
\end{rSubsection}

\begin{rSubsection}{LunarWSN}{Nov 2020 --- Aug 2021}{ }{MIT Media Lab}
\item LunarWSN is a Wireless Sensor Network node designed for In-Situ lunar water ice detection.
\item Designed a fully functional cubic sensor node prototype that can be ballistically deployed from a rover or lander to regions of interest that might be unsafe or impractical for rovers or landers. The node is a light ($<170g$), miniaturized ($5cm \times 5cm \times 5cm$), modular design, that allows sensor payloads to be customized to different scientific missions. As a representative case study, the node is equipped with an impedance sensor designed to measure the permittivity of the lunar soil, which infers water content.
\item Finished the system function tests (wireless localization, wireless communication, and sensing capability) in a lab environment. The work reached Technology Readiness Level 4 (TRL4).
\item Finished my Master thesis based on this work.
\end{rSubsection}

\begin{rSubsection}{WOSNA}{Aug 2020 --- Present}{ }{MIT Media Lab}
\item The WOSNA is short for Work Out on-body Sensor Network Assistant.
\item Designed an on-body sensor network that monitors workout performance. The sensor network is composed of multiple miniature sensor nodes, and each node is a tiny suction cup equipped with a pulse sensor and Bluetooth Low Energy communication. The sensor nodes can suck on the desired part of the body to monitor the performance of some specific muscle. The sensor nodes are very small and can adapt to the irregular body surface.
\end{rSubsection}

\begin{rSubsection}{PCBPT}{Aug 2019 --- Oct 2019}{ }{MIT Media Lab}
\item The PCBPT is a PCB automatic probe tester for in-circuit debugging.
\item Designed an automatic PCB probe system to help with PCB debugging.
With the help of PCBPT, users can choose desired signals in the schematic,
and the PCBPT will choose proper pads on the PCB for the selected signals
and place probes on the pads. All the users need to do are select signals
and check the waveforms on an oscilloscope.
\end{rSubsection}

\begin{rSubsection}{AMS-02 UTTPS ground test monitor and control software}{Apr 2018 --- Apr 2019}{ }{CERN (The European Organization for Nuclear Research)}
\item The UTTPS is the Upgraded Tracker Thermal Pump System of the Alpha Magnetic Spectrometer (AMS-02), and AMS-02 is a particle detection working on the international space station.
\item Developed the control and monitoring software for the thermal vacuum test of the AMS-02 (Alpha Magnetic Spectrometer, a state-of-the-art particle physics detector operating on the International Space Station) Upgraded Tracker Thermal Pump System (UTTPS). The software is designed by using labVIEW.
\item The UTTPS has been installed to the AMS-02 by the end of Jan 2020 through four spacewalks.
\end{rSubsection}
\end{rSection}
%----------------------------------------------------------------------------------------
%   Technical Experience 
%----------------------------------------------------------------------------------------

\begin{rSection}{publications}

\begin{itemlabel}
\item \textbf{Fangzheng Liu}, Ariel Ekblaw, Joseph Paradiso.
"LunarWSN node - a Wireless Sensor Network node designed for In-Situ lunar water ice detection."
SmallSat conference 2022 (Aug 2022). [Accepted]
\smallskip
\smallskip
\smallskip
\item B Haghighat, J Boghaert, Z Minsky-Primus, J Ebert, \textbf{F Liu}, M Nisser, A Ekblaw, and R Nagpal.
"An Approach Based on Particle Swarm Optimization for Inspection of Spacecraft Hulls by a Swarm of Miniaturized Robots."
13th International Conference on Swarm Intelligence (ANTS 2022). [Accepted]
\end{itemlabel}
\end{rSection}

%----------------------------------------------------------------------------------------
%   TECHNICAL STRENGTHS SECTION
%----------------------------------------------------------------------------------------

\begin{rSection}{Technical Skills}

\begin{itemlabel}

\item \textbf{Programming languages and related} --- C, Python, JavaScript, Git, MATLAB, LabVIEW, Arduino

\item \textbf{Computer aided design/engineering} --- Altium Designer, KiCAD, EAGLE, Mentor Graphics PADS, SolidWorks, Fusion 360, 3D printing.

\end{itemlabel}
\end{rSection}

%----------------------------------------------------------------------------------------
%   Teachings and ACTIVITIES
%----------------------------------------------------------------------------------------
\begin{rSection}{Teachings and ACTIVITIES}

\begin{itemlabel}

\item \textbf{Head Teaching Assistant} (2022 Spring) --- MIT course "Sensors for Interactive Environments."

\item \textbf{Teaching Assistant} (2021 Spring) --- MIT course "Adventures in Sensing."

\end{itemlabel}
\end{rSection}

%----------------------------------------------------------------------------------------
%   AWARDS
%----------------------------------------------------------------------------------------
\begin{rSection}{AWARDS}

\begin{itemlabel}

\item \textbf{China National Scholarship} (2011) --- Top 0.2\%
\item \textbf{Intel Cup Undergraduate Electronic Design Contest --- Embedded System Design Invitation Contest} (2014) --- Second prize
\item \textbf{Angela Leong Fund Fellowship} (2022-2023 academic year) --- 1/year in MIT

\end{itemlabel}
\end{rSection}

%----------------------------------------------------------------------------------------
%   INTERESTS
%----------------------------------------------------------------------------------------
\begin{rSection}{Interests}
Hiking\hspace{1cm}Cycling\hspace{1cm}Basketball\hspace{1cm}CS go\hspace{1cm}Electronics Hobbyist
\end{rSection}

%----------------------------------------------------------------------------------------

\end{document}
