%%%%%%%%%%%%%%%%%%%%%%%%%%%%%%%%%%%%%%%%%
% Medium Length Professional CV
% LaTeX Template
% Version 2.0 (8/5/13)
%
% This template has been downloaded from:
% http://www.LaTeXTemplates.com
%
% Original author:
% Trey Hunner (http://www.treyhunner.com/)
%
% Important note:
% This template requires the resume.cls file to be in the same directory as the
% .tex file. The resume.cls file provides the resume style used for structuring the
% document.
%
%%%%%%%%%%%%%%%%%%%%%%%%%%%%%%%%%%%%%%%%%

%----------------------------------------------------------------------------------------
%   PACKAGES AND OTHER DOCUMENT CONFIGURATIONS
%----------------------------------------------------------------------------------------

\documentclass{resume} % Use the custom resume.cls style
\usepackage[left=0.7in,top=0.4in,right=0.7in,bottom=0.5in]{geometry} % Document margins
\usepackage{hyperref}
\usepackage{xcolor}
%----------------------------------------------------------------------------------------
%   DOCUMENT START
%----------------------------------------------------------------------------------------
\name{Fangzheng Liu} % Your name
\address{\href{mailto:fzliu@mit.edu}{fzliu@mit.edu} \\ \href{https://github.com/Ironman-007}{Fangzheng's github} \\ \href{https://ironman-007.github.io/FZ-portfolio/index.html}{Portfolio}} % Your phone number and email
\begin{document}
\title{Fangzheng Resume}
%----------------------------------------------------------------------------------------
%   EDUCATION SECTION
%----------------------------------------------------------------------------------------
% \begin{rSection}{Summary}
% \begin{itemlabel}
% \item Main research is about miniature robotic and wireless sensor networks for planetary explorations.
% One of my designed tiny robots will be sent to the moon in 2024.
% \item Side projects include human-computer interaction design and on-body sensor networks.
% \item Background in electronics engineering, signal processing, mechanical system design, and so on.
% \end{itemlabel}
% \end{rSection}

\begin{rSection}{Proof of excellence}
      One of my designed miniature robot (3 cm $\times$ 3 cm $\times$ 3 cm), which I named the "AstroAnt",
      has been sent to the Moon in Feb 2025 in the Intuitive Machines IM-2 Lunar south pole mission.
      It is the smallest rover and the first Bluetooth device that has ever been sent to the Moon.
      My PhD research is about miniature robotic and wireless sensor networks for future planetary in-situ exploration.
\end{rSection}

% \begin{rSection}{Fun fact about me}
%       A printmaking I made when I was ten years old was collected by the Muroran City Museum in Japan.
% \end{rSection}

\begin{rSection}{Education}

\begin{rSubsection}{Massachusetts Institute of Technology}{September 2021 --- May 2026 (expected)}{Ph.D. in Media Arts and Sciences}{Responsive Environments Group, MIT Media Lab}
%\vspace{-.1cm}
%Overall GPA: 78\%
\vspace{-.65cm}
\item[]
\end{rSubsection}

\begin{rSubsection}{Massachusetts Institute of Technology}{September 2019 --- Aug 2021}{M.S. in Media Arts and Sciences}{Responsive Environments Group, MIT Media Lab}
%\vspace{-.1cm}
%Overall GPA: 78\%
\vspace{-.65cm}
\item[]
\end{rSubsection}

\begin{rSubsection}{Beijing Institute of Technology}{September 2015 --- Apr 2018}{M.S. in Information and Communication Engineering}{}
%\vspace{-.1cm}
%Overall GPA: 78\%
\vspace{-.65cm}
\item[]
\end{rSubsection}

\begin{rSubsection}{Beijing Institute of Technology}{September 2011 --- Jun 2015}{B.S. in Information Engineering}{}
%\vspace{-.1cm}
%Overall GPA: 78\%
\vspace{-.65cm}
\item[]
\end{rSubsection}

\end{rSection}

%----------------------------------------------------------------------------------------
%   WORK EXPERIENCE SECTION
%----------------------------------------------------------------------------------------
\begin{rSection}{WORK EXPERIENCE}
\begin{rSubsection}{Engineer (supervisor: Prof. Samuel C.C. Ting (1976 Nobel laureate in physics))}{}
      {CERN (the European Organization for Nuclear Research)}
      {Apr 2018 --- Apr 2019}
\item Designed the ground test monitor and control software for the UTTPS (Upgraded Tracker Thermal Pump System) of the AMS-02
      (Alpha Magnetic Spectrometer, a high-energy particle Spectrometer operating on the International Space Station).
\item The UTTPS has been installed to the AMS-02 by the end of Jan 2020 through four spacewalks.
\end{rSubsection}
\end{rSection}

\begin{rSection}{ACADEMIC RESEARCH EXPERIENCE}
\begin{rSubsection}{AstroAnt - my designed robot is on the Moon!}{Jan 2021 --- Mar 2023}{ }{MIT Media Lab}
    \item Lead engineer of the MIT Media Lab AstroAnt Lunar mission.
    \item Developed a miniature robot (the "AstroAnt") to work on the top panel of the Lunar rover MAPP-1 (developed by the Lunar Outpost)
          to collect thermal data.
    \item Finished four parabolic flights, a total of 120 parabolas.
    \item Finished all space-grade tests for the AstroAnt, and achieved the TRL (Technology Readiness Levels) 8 level.
    \item The AstroAnt has been sent to the Lunar South Pole in the IM-2 mission (Intuitive Machines-2) in March 2025.
\end{rSubsection}

\begin{rSubsection}{HexSense}{May 2023 --- Present}{ }{MIT Media Lab}
\item Developed the "HexSense", which is a miniature wireless sensor node that can be
ballistically deployed from rovers/landers or dropped from a fly-by satellite to the Lunar surface.
After deployment, the HexSense can automatically stand up right and start collecting scientific data.
\item Finished all the system function tests in a lab environment and achieved the TRL 4 level.
\item Finished a 10-day field expedition in Svalbard in July 2023.
One HexSense was deployed to collected local environmental data.
\item Finished a 7-day field expedition in the lava tubes on the Canary Island in Feb 2024.
Two HexSense were deployed to study the shielding protection of the lava tubes.
\item Finished one parabolic flight in May 2024 and proved self-orientation capability in Lunar gravity.
\end{rSubsection}

\begin{rSubsection}{Lunar Tumbleweed}{May 2025 --- Present}{ }{MIT Media Lab}
\item The Lunar Tumbleweed is a movable wireless sensor node that is designed to work on the Moon.
\item The Lunar Tumbleweed can realize electrical power-free mobility on the lunar surface through
the sun iradiance and custom-designed actuators driven by custom-made shape memory alloy spring.
\end{rSubsection}

\begin{rSubsection}{LunarWSN (Master thesis at MIT)}{Nov 2020 --- Aug 2021}{ }{MIT Media Lab}
\item Developed a miniature cubic wireless sensor node, that we named "LunarWSN",
      that can be ballistically deployed from rovers/landers or dropped from a fly-by satellite to the Lunar surface.
\item The LunarWSN node is designed for In-Situ lunar water ice detection.
\item Finished all the system function tests in a lab environment and achieved the TRL 4 level.
\end{rSubsection}

\begin{rSubsection}{LunarDeltaT}{Feb 2023 --- Jun 2023}{ }{MIT Media Lab}
\item Developed an approach to harvest energy through temperature gradient across
      the surface of the HexSense when it's working on the Moon.
\item Finished all tests in a lab environment.
      LunarDeltaT shows an advantage over solar cells in a dusty environment (e.g., the lunar surface).
\end{rSubsection}

\begin{rSubsection}{CircuitScout (side project)}{Jun 2023 --- Oct 2023}{ }{MIT Media Lab}
\item Designed a bench-top machine tool to streamline the in-circuit debugging.
\item With the help of the PCBPT, when debugging PCB,
all the users need to do are selecting signals and checking the output without the need to do anything else.
\end{rSubsection}

\begin{rSubsection}{IO-Touch (side project)}{Oct 2023 --- Nov 2023}{ }{MIT Media Lab}
\item A pure software approach that can turn almost every GPIO into a capacitive sensor.
\end{rSubsection}

\begin{rSubsection}{Mind Cube (side project)}{Jul 2024 --- Aug 2024}{ }{MIT Media Lab}
\item Designed a miniature (3 cm x 3 cm x 3 cm) fidget cube toy that can collect data from all the sensors when it is
played in a hand. The data collected can be used to study one's real-time emotion.
The Mind Cube can also be used as a music/game controller.
\end{rSubsection}

\end{rSection}
%----------------------------------------------------------------------------------------
%   Technical Experience 
%----------------------------------------------------------------------------------------

\begin{rSection}{publications}
\begin{itemlabel}
\item \textbf{\textit{Fangzheng Liu}}, Nicolas STAS, Ariel Ekblaw, and Joseph A Paradiso.
"HexSense Lunar Mapping: Deployable 360 Cameras for Panoramic Inspection \& 3D Reconstruction".
In The 47th International IEEE Aerospace Conference (2026, submitted).
\smallskip

\item Leonie Bensch, Cody Paige, Don D. Haddad, \textbf{\textit{Fangzheng Liu}},
Nathan Perry, Gerrit Olivier, Jessica Todd, Joseph A. Paradiso
"Creating Immersive Digital Twins of Terrestrial Planetary Analogs with Multimodal Sensing and Game Engines for Virtual Exploration".
In The IEEE Pervasive Computing Special Issue - Defining A New Cross Reality: Digital Twins and Mixed Reality Worlds (2025, accepted).
\smallskip

\item \textbf{\textit{Fangzheng Liu}}, Kerri Cahoy, Ariel Ekblaw, and Joseph A Paradiso.
"A Wireless Lunar Sensor Node Powered by Temperature Gradients across the Device's Surface".
In The 45th International IEEE Aerospace Conference (2024).
\smallskip

\item \textbf{\textit{Fangzheng Liu}}, Nathan Perry, Ariel Ekblaw, and Joseph A Paradiso, 2025.
"A Field Expedition and Parabolic Flight Experiment for the HexSense: A Type of Ballistic
Deployed Self-Oriented Wireless Sensor Nodes for Future Lunar Exploration".
In AIAA Scitech 2025 Forum.
\smallskip

\item \textbf{\textit{Fangzheng Liu}},
Cody Paige, Ariel Ekblaw, and Joseph A Paradiso.
"HexSense: Self-Oriented Ballistic Deployed Wireless Sensor Nodes for Lunar Exploration".
In Accelerating Space Commerce, Exploration, and New Discovery (ASCEND) 2024.
\smallskip

\item \textbf{\textit{Fangzheng Liu}}, Ariel Ekblaw, Joseph Paradiso.
"LunarWSN node - a Wireless Sensor Network node designed for In-Situ lunar water ice detection."
SmallSat conference 2022 (Aug 2022).
\smallskip

\item \textbf{\textit{Fangzheng Liu}}, Blanchard, L., Haddad, D.D. and Paradiso, J.A., 2025.
"Two Sonification Methods for the MindCube." arXiv preprint arXiv:2506.18196.
\smallskip

\item \textbf{\textit{Fangzheng Liu}} and Joseph A Paradiso. 2023.
PrintedCircuit Board (PCB) Probe Tester (PCBPT) - a Compact Desktop system that Helps with Automatic PCB Debugging.
In The 36th Annual ACM Symposium on User Interface Software and Technology (UIST '23 Adjunct),
October 29--November 01, 2023, San Francisco, CA, USA. ACM, New York, NY, USA 3 Pages. https://doi.org/10.1145/3586182.3615800.
\smallskip

\item \textbf{Liu, F.}, Dementyev, A., Wicaksono, I. and Paradiso, J.A., 2025, September.
Experiencing EmbedNet: Embedding self-sensing to 3D casting objects.
In Adjunct Proceedings of the 38th Annual ACM Symposium on User Interface Software and Technology (pp. 1-4).
\smallskip

\item Ariel Ekblaw, Juliana Cherston, \textbf{\textit{Fangzheng Liu}}, Irmandy Wicaksono, Don Derek Haddad, Valentina Sumini, Joseph A. Paradiso.
"From UbiComp to Universe - Moving Pervasive Computing Research Into Space Applications."
IEEE Pervasive Computing 2022.
\smallskip

\item B Haghighat, J Boghaert, Z Minsky-Primus, J Ebert, \textbf{\textit{Fangzheng Liu}}, M Nisser, A Ekblaw, and R Nagpal.
"An Approach Based on Particle Swarm Optimization for Inspection of Spacecraft Hulls by a Swarm of Miniaturized Robots."
In 13th International Conference on Swarm Intelligence (ANTS 2022).
\smallskip

\item \textbf{\textit{Fangzheng Liu}}, Haddad, D.D. and Paradiso, J., 2024, October.
MindCube: an Interactive Device for Gauging Emotions.
In Adjunct Proceedings of the 37th Annual ACM Symposium on User Interface Software and Technology (pp. 1-2).
\smallskip

\item LUO Qing-sheng, ZHOU Chen-yang, JIA Yan, GAO Jian-feng, \textbf{\textit{Fangzheng Liu}}:
"CPG-Based Control Scheme for Quadruped Robot to Withstand the Lateral Impact."
2015. Journal of Beijing Institute of Technology, 35(4), pp.384-390.
\end{itemlabel}
\end{rSection}

%
\begin{rSection}{Patents}

\begin{itemlabel}
\item CUI Wei, HOU Jian-gang, \textbf{\textit{Fangzheng Liu}}, SHEN Qing, XIANG Jing-zhi, WU Si-liang: "A Radar Echo Delay Coherent Simulation Method Based on Digital Radio Frequency Signal Storage." Chinese patent: 2017104551967 (G01S7/40). Filed on Jun 16, 2017, and issued on Dec 18, 2018. \href{http://www.zlqiao.com/zlqiao/patent-f0301af6125548659bae9e05ed9543d6.html}{[LINK]}\\
(Advisors: CUI Wei, HOU Jian-gang)
\smallskip
\smallskip
\smallskip
\item CUI Wei, SHEN Qing, HOU Jian-gang, \textbf{\textit{Fangzheng Liu}}, XIANG Jing-zhi, WU Si-liang: "A Doppler Frequency Coherent Simulation Method for Radar Echoes Based on Real-time Frequency Measurement." Chinese patent: 2017104552014 (G01S7/40). Filed on Jun 16, 2017, and issued on Oct 9, 2018. \href{http://www.zlqiao.com/zlqiao/patent-4dc7dd85795d40a08320e507561834ca.html}{[LINK]}\\
(Advisors: CUI Wei, HOU Jian-gang)
\end{itemlabel}
\end{rSection}

%

%
\begin{rSection}{Selected Media reports}

\begin{itemlabel}
\item \textbf{Fast Company -} "MIT designed these tiny R2-D2 robots to help keep spaceships safe" \href{https://www.fastcompany.com/91194519/mit-designed-these-tiny-r2-d2-robots-to-help-keep-spaceships-safe}{[LINK]}
\item \textbf{Forbes - }"MIT Will Return To The Moon For The First Time Since Apollo, Thanks To This Space Startup" \href{https://www.forbes.com/sites/ariannajohnson/2022/11/09/mit-will-return-to-the-moon-for-the-first-time-since-apollo-thanks-to-this-space-startup/?sh=45edf5d86d72}{[LINK]}
\item \textbf{MIT News - } "MIT engineers prepare to send three payloads to the moon" \href{https://news.mit.edu/2025/mit-engineers-prepare-send-three-payloads-moon-0225}{[LINK]}
\item \textbf{Create the Future 2024 Design Contest} - \textcolor{red}{Category winner (Robotics)} "Astroant: a Miniature Symbiotic Robotic Serving on the Outside Surfaces of Spacecraft, Rovers, and Landers for Inspection and Diagnostic Tasks" \href{https://contest.techbriefs.com/2024/entries/robotics-and-automation/13181-0707-232212-astroant-a-miniature-symbiotic-robotic-serving-on-the-outside-surfaces-of-spacecraft-rovers-and-landers-for-inspection-and-diagnostic-tasks}{[LINK]}
\item \textbf{Castrol - } "Castrol membership supports the astroant payload program" \href{https://www.castrol.com/en/global/corporate/about-castrol/newsroom/mit-lunar-landings.html}{[LINK]}
\item \textbf{MIT Media Lab - } "Media Lab + Castrol Collaboration: Meet HexSense" \href{https://www.youtube.com/watch?v=wPBvrFLA74g}{[LINK]}
\item \textbf{hackster.io - } "IoT Shoots for the Moon" \href{https://www.hackster.io/news/iot-shoots-for-the-moon-3e8b97780c54}{[LINK]}
\item \textbf{Create the Future 2023 Design Contest} - \textcolor{red}{Top 100 Entry (Electronics)} "PCB Probe Tester (PCBPT) - a Compact Desktop System that Helps with Automatic PCB Debugging" \href{https://contest.techbriefs.com/2023/entries/electronics/12473-0701-202435-pcb-probe-tester-pcbpt-a-compact-desktop-system-that-helps-with-automatic-pcb-debugging}{[LINK]}
\item \textbf{Hackaday - } "Hackaday Prize 2023: Circuit Scout Lends A Hand (Or Two) For Troubleshooting" \href{https://hackaday.com/2023/08/11/hackaday-prize-2023-circuit-scout-lends-a-hand-or-two-for-troubleshooting/}{[LINK]}
\item \textbf{hackster.io - } "CircuitScout Aims to Automate the Painful Process of Probing Test Pads on Your PCBs" \href{https://www.hackster.io/news/circuitscout-aims-to-automate-the-painful-process-of-probing-test-pads-on-your-pcbs-4f62c34249ee}{[LINK]}

\end{itemlabel}
\end{rSection}

%----------------------------------------------------------------------------------------
%   AWARDS
%----------------------------------------------------------------------------------------
\begin{rSection}{AWARDS}
\begin{itemlabel}
\item \textbf{Amazon Robotics PhD communication competition (3rd place)} (2025)
\smallskip
\smallskip

\item \textbf{The Create the Future Design Contest category winner (Robotics and Automation)} (2024-2025 academic year)
\smallskip
\smallskip

\begin{itemlabel}
\item \textbf{Harold Horowitz (1951) Student Research Fund award} (2023-2024 academic year)
\smallskip
\smallskip

\item \textbf{Angela Leong Fund Fellowship} (2022-2023 academic year) - 1 student/year in MIT
\smallskip
\smallskip

\item \textbf{Intel Cup Undergraduate Electronic Design Contest - Embedded System Design Invitation Contest} (2014) - Second prize
\smallskip
\smallskip

\item \textbf{China National Scholarship} (2011) - Top 0.2\%
\smallskip
\smallskip

\end{itemlabel}
% \end{rSection}
\end{itemlabel}
\end{rSection}

%----------------------------------------------------------------------------------------
%   TECHNICAL STRENGTHS SECTION
%----------------------------------------------------------------------------------------

\begin{rSection}{Technical Skills}

\begin{itemlabel}

\item \textbf{Electronics} - Embedded system (nRF52, STM32, Atmel ATSAMD, Atmel AVR, ESP32), FPGA (Xlinix 7 series), analog circuit
\item \textbf{Programming languages and related} - C, C++, VHDL, Python, JavaScript, Git, LabVIEW, Arduino
\item \textbf{Computer-aided design/engineering} - Altium Designer, KiCAD, EasyEDA, Mentor Graphics PADS, SolidWorks, Fusion 360, Inventor, Onshape.
\item \textbf{Manufacturing skills} - 3D modeling/printing (FDM, SLA, SLS), Molding \& Casting, CNC machining, PCB milling, PCB soldering, Laser cutting, water jet cutting.
\item \textbf{Space grade tests} - thermal vacuum test.
\item \textbf{Others} - Parabolic flight certificates !!! :]
\end{itemlabel}
\end{rSection}


%----------------------------------------------------------------------------------------
%   INTERESTS
%----------------------------------------------------------------------------------------
\begin{rSection}{Teachings and ACTIVITIES}

\begin{itemlabel}
\item \textbf{Teaching Assistant} (2024 Fall) - MIT "MAS.863/4.140/6.9020 How To Make (almost) Anything"
\smallskip
\smallskip

\item \textbf{Teaching Assistant} (2023 Fall) - MIT "MAS.863/4.140/6.9020 How To Make (almost) Anything"
\smallskip
\smallskip

\item \textbf{Head Teaching Assistant} (2022 Spring) - MIT "MAS.836 Sensing Technologies for Interact Environments"
\smallskip
\smallskip

\item \textbf{Teaching Assistant} (2021 Spring) - MIT "MAS.S76 Adventures in Sensing"

\end{itemlabel}
\end{rSection}

%----------------------------------------------------------------------------------------
%   INTERESTS
%----------------------------------------------------------------------------------------
\begin{rSection}{Interests}
Painting\hspace{0.5cm}Hiking\hspace{0.5cm}Cycling\hspace{0.5cm}rock climbing\hspace{0.5cm}Basketball\hspace{0.5cm}Electronics
\end{rSection}

%----------------------------------------------------------------------------------------

\end{document}
